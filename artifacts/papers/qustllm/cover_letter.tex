\documentclass[10pt,stdletter,dateno]{newlfm}
\usepackage{kpfonts}
\usepackage{url}

\widowpenalty=1000
\clubpenalty=1000

\newlfmP{headermarginskip=10pt}
\newlfmP{sigsize=10pt}
\newlfmP{dateskipafter=10pt}
\newlfmP{addrfromphone}
\newlfmP{addrfromemail}
\PhrPhone{Phone}
\PhrEmail{Email}

\namefrom{Chao Hui Huang (Ph.D.)\\Senior Principal Image Analyzer\\Pfizer Oncology Division}
\addrfrom{%
	\today\\[10pt]
	Pfizer Inc.\\
	19646 Science Dr., La Jolla\\
	San Diego, CA 92121
}
\phonefrom{650-430-0719}
\emailfrom{chao-hui.huang@pfizer.com, huangch.tw@gmail.com}

\addrto{Editor-in-Chief\\Precision Oncology\\Nature Publishing Group}

\greetto{Dear Editor-in-Chief,}
\closeline{Sincerely,}
\begin{document}
	\begin{newlfm}
I am writing to submit our manuscript entitled "QuST-LLM: Integrating Large Language Models for Comprehensive Spatial Transcriptomics Analysis" for consideration for publication in Precision Oncology.

In this paper, we present QuST-LLM, a novel enhancement of QuPath that leverages the power of large language models (LLMs) to analyze and interpret spatial transcriptomics (ST) data. In addition to simplifies the complex and high-dimensional ST data by providing a complete workflow, QuST-LLM is able to convert intricate ST data into human languages based on gene ontology annotations, thereby greatly enhancing the interpretability of ST data. 
%As a result, QuST-LLM equips researchers with a powerful tool to decode the spatial and functional complexities of tissues. 
Further, by establishing this "translator" between ST data and human languages, QuST-LLM enables direct mapping from existing ST data to hundreds of thousands of published biomedical-related articles. As a result, QuST-LLM represents an opportunity to promote new discoveries and progress in biomedical research.

Our experimental results suggest the robust ability of QuST-LLM to interpret biological spatial patterns and identify specific single-cell clusters or regions based on user-provided descriptions in natural language. We believe that our findings provide researchers with a potent functionality to unravel the spatial and functional complexities of tissues, fostering novel insights and advancements in biomedical research.

We consider that our paper is a good fit for Precision Oncology because it offers a novel approach to spatial transcriptomics analysis, a topic of increasing interest in the field of genomics. We believe our findings could be of great interest to your readership, contributing to the ongoing discourse on the application and potential of large language models in computational biology.

All authors have approved the manuscript and agree with its submission to Precision Oncology. This is an original work that has not been submitted to or published in any other journal. a preprint has been released at \url{https://arxiv.org/abs/2406.14307}. There are no conflicts of interest to declare.

Thank you for considering our manuscript for publication. We look forward to your positive response.
		
	\end{newlfm}
\end{document}