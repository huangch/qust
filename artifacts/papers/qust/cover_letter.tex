\documentclass[10pt,stdletter,dateno]{newlfm}
\usepackage{kpfonts}
\usepackage{url}

\widowpenalty=1000
\clubpenalty=1000

\newlfmP{headermarginskip=10pt}
\newlfmP{sigsize=10pt}
\newlfmP{dateskipafter=10pt}
\newlfmP{addrfromphone}
\newlfmP{addrfromemail}
\PhrPhone{Phone}
\PhrEmail{Email}

\namefrom{Chao Hui Huang (Ph.D.)\\Senior Principal Image Analyzer\\Pfizer Oncology Division}
\addrfrom{%
	\today\\[10pt]
	Pfizer Inc.\\
	19646 Science Dr., La Jolla\\
	San Diego, CA 92121
}
\phonefrom{650-430-0719}
\emailfrom{chao-hui.huang@pfizer.com, huangch.tw@gmail.com}

\addrto{Editor-in-Chief\\Precision Oncology\\Nature Publishing Group}

\greetto{Dear Editor-in-Chief,}
\closeline{Sincerely,}
\begin{document}
	\begin{newlfm}
		I am writing to submit our manuscript entitled "QuST: QuPath Extension for Integrative Whole Slide Image and Spatial Transcriptomics Analysis" for consideration for publication in Precision Oncology. We believe that our work addresses a critical gap in the field of digital pathology by presenting a novel approach for integrating deep learning (DL)-based whole slide image (WSI) analysis with spatial transcriptomics (ST) analysis.
		
		The manuscript describes the development of QuST, a QuPath extension that aims to bridge the gap between H\&E WSI and ST analysis tasks. We highlight the importance of integrating deep learning-based WSI analysis and ST analysis in understanding disease biology, and discuss the challenges associated with integrating these modalities due to differences in data formats and analytical methods. We believe that QuST has the potential to unlock valuable information hidden within WSI data, which can greatly support and enhance ST analysis.
		
		Our manuscript contributes to the novelty on the integration of DL-driven WSI analysis and ST analysis, which are both rapidly advancing fields in precision oncology. By leveraging the power of deep learning, QuST expands the field of view for histopathological image analysis, while also providing a means to bridge the gap between tissue spatial analysis and biological signals. Furthermore, we provide the QuST source code on GitHub and comprehensive documentation to facilitate its adoption and further development by the research community.
		
		We believe that Precision Oncology is an ideal platform to disseminate our results, as it is a highly regarded journal that focuses on cutting-edge research in the field of oncology. Our manuscript aligns with the journal's scope and aims to contribute to the advancement of precision medicine in oncology.
		
		Thank you for considering our manuscript for publication. We appreciate the time and effort invested by the editorial team and reviewers in evaluating our work. We look forward to hearing from you soon.
		
	\end{newlfm}
\end{document}